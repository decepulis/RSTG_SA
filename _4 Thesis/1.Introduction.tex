% - Welcome to Information Foraging
If we understand how a user seeks information, then we can optimize an information environment to make that information easier to retrieve. Pirolli and Card worked to understand information-seeking by defining information foraging theory~\cite{ift}.
Information foraging theory describes a user's information search by equating it to nature's optimal foraging theory: in the same way that scent carries a predator to a patch where it may find its prey, a user follows cues in their environment to information patches where they might find their information.

% - Patches Patches Patches Patches
Information foraging theory has seen many applications since Pirolli's seminal work. For example, in web search, foragers follow information scent to their patches, web pages~\cite{pirolliWeb,wufis}. Understanding how foragers find information in web search has helped developers design the information environment of their web pages~\cite{wufis}. In code navigation and debugging, information foraging theory describes how developers seek to resolve a bug report by navigating from fragment to fragment of code to define and fix the problem~\cite{navValueCost}.
By understanding the process of finding and fixing bugs, models can be written to assist in this process~\cite{pfisRevisit}. In both of these scenarios, the patch is clearly defined: in web search, a forager's patch is a web page, and in debugging, a developer's patch might be a fragment of code. What happens, though, when the patch is not clearly defined?

% - We don't have patches in socio-technical systems
Consider socio-technical systems, where information artifacts are connected to people. Facebook, YouTube, Twitter, GitHub, and Wikipedia all have information artifacts, like posts and code snippets, with a rich context of social interactions tying them together. A forager traverses both the artifacts and the social structures behind them in an information seeking task. Therefore, both artifacts and social structures should be considered when defining a patch, and patches are not necessarily immediately evident. Consider a user wondering how a reposted image became so popular among their friends: how could patches be defined for this forager? Photos, posts, comments, friends' connections? A GitHub code fragment or Facebook Friend can't answer these questions alone, so patches must consist of some combination. In this paper, we describe a method for delineating such patches.

% - RT is a socio-technical system! RT is important
Requirements traceability is an ideal field for examining patch creation in a socio-technical environment. Requirements traceability is a socio-technical system used to describe and follow the life of a requirement by examining the trail of artifacts and people behind them, from the requirement's inception to implementation. Requirements traceability problems, as studied by Gotel and Finkelstein~\cite{ICSE30}, arise when questions about the production and refinement of requirements cannot be answered. More specifically, with a traceability failure, US Food and Drug Administration might cast doubt in product safety~\cite{ICSE46}. With a traceability failure, the CEO of a prominent social media company cannot explain to Congress how a decision to withhold information from customers was made~\cite{politicoFacebook}. Applying information foraging to these problems could significantly increase efficiency and efficacy of these traceability tasks.

% - Start defining RT in terms of foraging. patch patch patch patch.
We relate requirements traceability to information foraging theory and its patches by considering requirements traceability questions. We define a requirements traceability question as a query that a project stakeholder issues \textit{in situ} wanting to know a requirement's life. A requirements traceability question is where a user's traceability task becomes a foraging task; the question represents the user's information need, or foraging prey. If questions represent a traceability forager's prey, what represents a traceability forager's patch? Put simply, we aim to answer the research question: \textit{where should a user search to understand their requirements traceability question?}

% - Our contribution
This paper makes two contributions by deriving a method for delineating these patches. First, by examining requirements socio-technical graphs constructed from four requirements repositories containing 125 traceability questions, we identify classes of relationships that should be considered in similar requirements traceability tasks. Second, we derive an algorithm, based on spreading activation, which combines these classes with information foraging concepts to create relevant patches where foragers can conduct their traceability tasks. The patches that our algorithm produces are as small as 5-10 nodes\textemdash a manageable quantity for a forager\textemdash representing knowledgeable users and useful information artifacts. The method for identifying these classes and deriving this algorithm can be extended to other socio-technical tasks.