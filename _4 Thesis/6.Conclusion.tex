\section{Implications}
%We discuss in this section the implications of our work as well as the limitations of our study.
Piorkowski and his colleagues~\cite{navValueCost} codified the fundamental challenges faced by software developers when foraging in the information environment. We believe that our socio-technical approach can help directly address the challenge of ``prey in pieces'' where the foraging paths were too long and disconnected by different topologies. By explicitly integrating humans in the underlying topology, information foragers can exploit a richer set of relationships.

Codebook~\cite{codebook10} confirmed that the small-world phenomenon~\cite{Chakrabarti-CSUR06} was readily observed in the socio-technical networks built from the software repositories, e.g., any pair of stakeholders would be connected in six-or-less hops. Our findings suggested that the requirements-centric socio-technical graphs are even smaller with relevant nodes surrounded in four-or-less degrees of separation from the traceability forager's question. Meanwhile, our results revealed several common relationships and their compositions. In light of the recent work on collecting practitioners' natural-language requirements queries (e.g.,~\cite{Pruski-REJ15, Lohar-REFSQ16, Malviya-RE17}), the patterns uncovered by our study could be used to better classify and answer project stakeholders' traceability needs.

Automated requirements traceability tools have been built predominantly by leveraging text retrieval methods~\cite{ICSE15}. These tools neglect an important factor\textemdash familiarity\textemdash which plays a crucial role in tracking the life of a requirement. Our work further points out that familiarity is multi-faceted: it could be the person familiar with the issue, the project, or the asker that is knowledgeable about the traceability information. Our results here are to be contrasted with the empirical work carried out by Dekhtyar \emph{et al.}~\cite{Dekhtyar-RE11} showing that experience (e.g., years worked in software industry) had little impact on human analysts' tracing performance. While a developer's overall background may be broad, we feel that the specific knowledge about the subject software system and the latent relationships established with project stakeholders do play a role in requirements tracing. Automated ways of inferring a developer's knowledge degree (e.g.,~\cite{Fritz-TOSEM14}) would be valuable when incorporated in traceability tools.

Requirements traceability serves as a critical case~\cite{yin03} for our investigation into developers' information foraging in a socio-technical environment. This is because, without the traceability information, many software engineering activities cannot be undertaken, such as verification that a design satisfies the requirements, validation that requirements have been implemented, change impact analysis, system level test coverage analysis, and regression test selection~\cite{Hayes-TSE06}. However, requirements traceability is by no means the only case. In fact, due to information foraging theory's parsimony, its constructs (e.g., patch and scent) have been adapted to support a variety of tasks including debugging, refactoring, and software reuse~\cite{Fleming-TOSEM13, Ragavan-CHI16, Ragavan-CHI17}. The socio-technical patch created by our spreading activation algorithm, therefore, could help developers answer their needs beyond requirements traceability.

\section{Conclusion}
By considering common relationships connecting a requirements traceability question to its answer, and encoding these relationships into a spreading activation algorithm, we were able to delineate patches for use in understanding context surrounding requirements traceability questions in a socio-technical environment. In this process, we found that traceability questions were answered by users within four degrees of socio-technical separation; these users were typically Frequent Collaborators of the traceability forager or of the creator/assignee of the issue, or Frequent Contributors to the issue. Encoding these relationships as parameters to a spreading activation algorithm resulted in patches of nodes that a traceability forager could traverse, searching for their answer. While simply creating patches including all nodes within four degrees of socio-technical separation would include all answers, the addition of spreading activation created significantly smaller patches. We believe that this work can serve as a foundation for future work in creating patches in socio-technical networks